\section{Collections}\label{sec:collections}

In this section we will define the database structure. The following collections
are defined:
\begin{enumerate*}[label=]
	\item \code{Users};
	\item \code{AuthTokens};
	\item \code{Sources};
	\item \code{MarketData};
	\item \code{Strategies}.
\end{enumerate*}

Small changes may be made in future to the structure and the content of each
collection in order to ensure better performance or a simpler implementation of
the application.

\subsection{Users}

Each document represents a user.

\lstinputlisting[language=json, label={lst:userscollection},
caption={\code{Users} collection example.}]{users.json}

Fields' description:
\begin{description}
	\item[\_id] \textit{(string)} Username;
	\item[passwordHash] \textit{(string)} Password hash;
	\item[isAdmin] \textit{(boolean, optional)} Specifies whether the user
		is the administrator or not. If not present, is considered
		\code{false}.
\end{description}

\subsection{AuthTokens}

Each document represents an authorization token used to authenticate the user
with the server. Multiple clients can connect as the same user creating a
session (a token) for each client (this allows for a session management feature
to be implemented in future).

\lstinputlisting[language=json, label={lst:authtokenscollection},
caption={\code{AuthTokens} collection example.}]{authtokens.json}

Fields' description:
\begin{description}
	\item[\_id] \textit{(string)} Token hash (generated randomly). Admin
		tokens always start with a \code{0};
	\item[username] \textit{(string)} User's username;
	\item[expireTime] \textit{(ISODate)} Token expire time.
\end{description}

\subsection{Sources}

Each document represents a data source.

\lstinputlisting[language=json, label={lst:sourcescollection},
caption={\code{Sources} collection example.}]{sources.json}

Fields' description:
\begin{description}
	\item[\_id] \textit{(string)} Source name;
	\item[enabled] \textit{(boolean, optional)} Specifies if the market is
		enabled or not. If not present, is considered \code{false};
	\item[markets] \textit{(array, optional)} array of embedded documents
		that lists the market available for this source. The structure
		of these documents is the following:
		\begin{description}
			\item[id] \textit{(string)} Name of the market;
			\item[baseCurrency] \textit{(string)} Name of the quoted
				currency;
			\item[quoteCurrency] \textit{(string)} Name of the
				quoting currency;
			\item[granularity] \textit{(integer)} Minimum size,
				in minutes, of a candle;
			\item[selectable] \textit{(boolean, optional)} Specifies
				if users are allowed or not to test strategies
				on this market. If not present, is considered
				\code{false}. If \code{enabled} is \code{false},
				this is considered \code{false};
			\item[sync] \textit{(boolean, optional)} Specifies if
				the scraper should synchronize the data for this
				market with the data that it downloads from the
				source. If not present, is considered
				\code{false}. If \code{enabled} is \code{false},
				this is considered \code{false}.
		\end{description}
\end{description}

\subsection{MarketData}

Market data is bucketed into multiple documents.

\lstinputlisting[language=json, label={lst:marketdatacollection},
caption={\code{MarketData} collection example.}]{marketdata.json}

Fields' description:
\begin{description}
	\item[\_id] \textit{(ObjectId)} Unique ID of the document;
	\item[market] \textit{(string)} ID of the market. Format:
		\code{sourcename:marketname};
	\item[start] \textit{(ISODate)} The time (\code{t} field) of the first
		candle;
	\item[ncandles] \textit{(integer)} The number of candles in the
		\code{candles} array;
	\item[candles] \textit{(array)} Array of embedded documents where each
		document represent a trading day (candle). The structure of
		these documents is the following:
		\begin{description}
			\item[t] \textit{(ISODate)} The opening time of the
				day;
			\item[o] \textit{(numeric)} The opening price of the
				day;
			\item[h] \textit{(numeric)} The highest price of the
				day;
			\item[l] \textit{(numeric)} The lowest price of the day;
			\item[c] \textit{(numeric)} The closing price of the
				day;
			\item[v] \textit{(numeric)} The volume, \idest*{the
				amount traded during the day}.
		\end{description}
\end{description}

\subsection{Strategies}

Each document represents a strategy.

\lstinputlisting[language=json, label={lst:strategiescollection},
caption={\code{Strategies} collection example.}]{strategies.json}

Fields' description:
\begin{description}
	\item[\_id] \textit{(string)} Hash of the file that defines the
		strategy;
	\item[name] \textit{(string)} Name of the strategy;
	\item[author] \textit{(string)} Username of the user that submitted the
		strategy;
	\item[runs] \textit{(array, optional)} Array of embedded documents that
		list all the tests made by the users with this strategy. The
		structure of these documents is the following:
		\begin{description}
			\item[id] \textit{(ObjectId)} Unique ID of the run;
			\item[user] \textit{(string)} The user who launched the
				run;
			\item[parameters] \textit{(document)} Strategy's
				configuration (list of parameters):
				\begin{description}
					\item[market] \textit{(string)} Name of
						the market on which the strategy
						has been executed. Format:
						\code{sourcename:marketname};
					\item[inverseCross] \textit{(boolean,
						optional)} Specifies if the
						strategy has run on the inverse
						market cross for the selected
						market or not. If not present,
						is considered \code{false};
					\item[granularity] \textit{(integer)}
						Size of candles passed to the
						strategy (this value should be a
						multiple of the minimum
						granularity of data available);
					\item[startTime]
						\textit{(ISODate)} Time of the
						first data on which the strategy
						has been executed;
					\item[endTime]
						\textit{(ISODate)} Time of the
						last data on which the strategy
						has been executed;
					\item[\ldots] Other fields may be
						present (defined by the
						strategy).
				\end{description}
			\item[report] \textit{(document)} Report generated by
				the execution of the strategy:
				\begin{description}
					\item[netProfit] \textit{(numeric)} The
						profit (\(\interval[open
						right]{0}{+\infty}\)) or loss
						(\(\interval[open
						right]{-1}{0}\)) of the
						strategy, computed as the total
						profit minus the total loss of
						the trades (percentage of the
						initial amount);
					\item[grossProfit] \textit{(numeric)}
						The gross profit, computed as
						the total profits of all winning
						trades;
					\item[grossLoss] \textit{(numeric)} The
						gross loss, computed as the
						total losses of all losing
						trades;
					\item[hodlProfit] \textit{(numeric)} The
						profit (\(\interval[open
						right]{0}{+\infty}\)) or loss
						(\(\interval[open
						right]{-1}{0}\)) of the Buy and
						Hold strategy (percentage of the
						initial amount);
					\item[totalTrades] \textit{(integer)}
						The total number of trades;
					\item[openTrades] \textit{(integer)} The
						number of uncompleted trades;
					\item[winningTrades] \textit{(integer)}
						The number of winning trades;
					\item[maxConsecutiveLosing]
						\textit{(integer)} The maximum
						number of consecutive losing
						trades;
					\item[avgAmount] \textit{(numeric)} The
						average amount of money involved
						in trades (percentage of the
						initial amount);
					\item[avgDuration] \textit{(numeric)}
						The average duration, in terms
						of trading days, of a trade;
					\item[maxDrawdown] \textit{(numeric)}
						The maximum drawdown as defined
						in \chref{ch:specs}. This is
						expressed as a percentage of the
						initial amount;
					\item[aggregationTime]
						\textit{(integer)} The time, in
						seconds, needed for the
						execution of the aggregation
						pipeline defined
						in~\secref{subsec:marketdata-aggregations};
					\item[executionTime] \textit{(integer)}
						The time, in seconds, needed for
						the execution of the strategy.
				\end{description}
		\end{description}
\end{description}

Note that all amounts specified are expressed as a percentage of the initial
amount. This means that when a strategy opens a trade, it will specify a
percentage of the total invested amount. This allows the users to change the
initial amount for a strategy and view the report without the need to rerun the
strategy.
