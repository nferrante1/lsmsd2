\section{Browse strategies}\label{sec:browsestrategies}

When you select the ``Browse strategies'' menu entry, the system will ask for a
\emph{Strategy Name} to search. This can be any strategy available in the
system. You can write the full name of the strategy or just a part of it; you
can write it lower-case or upper-case; or you can just leave the field empty to
search for all available strategies.

When you press ENTER, a menu listing all the strategies that matches the search
criteria will be shown:

\begin{verbatim}
Search by Strategy Name (press ENTER to find all entities):

Strategy Name: Strategy

Select a strategy:
1)      Bollinger's Bands Strategy (by: user2)
2)      Golden Cross Strategy (by: user)
3)      MACD Strategy (by: user)
4)      RSI Strategy (by: user2)
5)      RSI Strategy v2 (by: admin)
0)      Go Back

Enter number:
\end{verbatim}

You can select a strategy by entering its number. If there are more than 20
strategies in the result set, the list will be paginated and two additional menu
entries will allow to navigate through the pages.

\subsection{View strategy}

After you select a strategy you can \standout{view the details} of the strategy;
\standout{browse its reports} of the executions made by all the users registered
in the system; \standout{run} the strategy over a market to generate a new
report; \standout{download} the strategy source code in a file;
\standout{delete} the strategy (if you are the creator of the strategy or the
administrator).

\begin{verbatim}
MACD Strategy | Select an action:
1)      View details
2)      Browse reports
3)      Run strategy
4)      Download strategy
5)      Delete strategy
0)      Go back

Enter number:
\end{verbatim}

\subsubsection{Browse reports}

When you select the ``Browse report'' menu entry, the application will ask to
search for a market\footnote{The search of a market follows the same rules of
the search of a strategy described before.}. You can then select a market to
show the reports of the executions over that market, or you can select ``All
Markets'' to show the reports for all the markets. Note that, if you are not the
administrator, you will see only the markets that the administrator has enabled
to be \emph{selectable}.

The you can select a report to show its details or delete it (if you are the
administrator or you have generated that report).

\begin{verbatim}
Report of MACD Strategy on COINBASE:BTC-EUR | Select an
  action:
1)      View report
2)      Delete report
0)      Go back

Enter number: 1

Define an amount:
Amount [100000.00]:
\end{verbatim}

When you choose to view a report, it will ask for an \emph{initial amount}. You
can just press ENTER to use the default value (shown between brackets) or set a
custom value. The report will be shown by adjusting all the values using the
initial amount specified.

\subsubsection{Run strategy}

When you select the ``Run strategy'' menu entry, the application will ask to
search (and then select) for the market on which you want to run the strategy.
Then it will ask if you want to run the strategy on the direct market cross or
in the inverted market, and the granularity, in minutes, of the candles (the
size of the trading day).

After that, depending on the chosen strategy, it may ask for additional
parameters to configure the strategy execution.

After the execution, the resulting report is shown.

\begin{verbatim}
Search by Market Name (press ENTER to find all entities):

Market Name: btc/eur

Select a market:
1)      BINANCE:BTC/EUR
2)      COINBASE:BTC/EUR
0)      Go Back

Enter number: 2

1) BTC/EUR (direct)
2) EUR/BTC (inverted)
Select cross: 1
Granularity [60]: 240
RSI Period (>0): 14
RSI Oversold (0-100): 30
RSI Overbrought (0-100): 70
Amount to Trade (0-1): 1

Initializing...done!
Running...   [#######################################] 100%
Generating Report...done!

Showing report with initial amount 1,000.00

... Additional output removed (Report) ...
\end{verbatim}

\subsubsection{Download strategy}

When you select the ``Download strategy'' menu entry, the application will ask
where to save the file. You can insert a relative or absolute path. The filename
must end with \code{.java}. Alternatively, you can just press ENTER to save the
strategy in the \code{strategy.java} file in the current directory (it will
overwrite the file if it is already present).
